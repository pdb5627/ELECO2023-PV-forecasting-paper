\section{Introduction}
\label{sec:intro}

\IEEEPARstart{O}{ver} recent decades, many factors have driven significant changes in the electrical power industry,
including climate change,
the growing use of renewable energy resources,
and ongoing rural electrification.
For agricultural consumers, electrification enables increased yields and economic development.
Ref. \cite{Mandelli2016} reviews technology and applications for off-grid systems for rural electrification.

For agricultural applications, photovoltaic (PV) energy is generally well aligned with water needs\cite{Aliyu2018}.
Ref. \cite{Aliyu2018} reviews numerous solar-powered water pumping applications around the world while \cite{Muhsen2017} reviews the PV and water pumping modeling, design, and control approaches in the literature.

One of the barriers to efficient use of PV energy in the grid and in micro-grids
is the difficulty of accurately predicting future energy availability.
As reviewed in \cite{Antonanzas2016}, many techniques have been developed for forecasting PV energy over various time horizons.
The authors of this paper have been working on techniques for optimal energy planning for agricultural microgrids.
For operational optimization, an effective day-ahead PV forecasting method was needed.
In recent years, LSTM-based neural networks have emerged as the state-of-the art
for day-ahead PV energy forecasting
\todo{I have a bunch of papers to look at for this, but I haven't read them yet. For space planning, I've just included just three citations, not necessarily the best ones.}
%\cite{Aillaud2020,Aslam2021,Gao2019,Gu2021,Liu2021,Luo2022,Mei2020,Qu2021,Wang2019,Wang2021,Wang2022,Yang2019,Yu2020,Zhang2020}.
\cite{Aillaud2020,Aslam2021,Gao2019}.

As promising as deep learning networks are for day-ahead forecasting,
they have the drawback of requiring 
a lot of historical data and significant computational resources for training.
\todo{You should look through the papers to see how much historical data they really need. Make sure this is actually affirmed in the literature before you state it.}
Therefore the authors developed methods for day-ahead PV forecasting with a design goal to be ``lightweight'', that is to require minimal weather forecast data, site-specific modeling data, historical generation data, and computational resources.
The target computing platform for the developed method is a single-board computer such as the Raspberry Pi or similar.
