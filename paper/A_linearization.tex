\numberwithin{equation}{section}
\appendix[Linearization Formulation]
\label{sec:linearization}
\renewcommand{\theequation}{A.\arabic{equation}}

Some non-linear functions were linearized using auxiliary variables to create linearized versions suitable for application in a MILP solver. Details of the formulations of these are provided in this appendix.
The linearized formulations generally follow the ideas from \cite{YALPMIP_logic} but are not identical in every case.

\subsection{max}

The $\max$ function was implemented using Big M and an additional binary variable $u_i$ for each of the input quantities $a_i$ to the function. The binary variable $u_i$ takes a value of \num{1} when $a_i$ is the maximum input value. A new variable $z$ is introduced as the output of the $\max$ function.

\begin{gather}
a_i \le z \le a_i + M (1 - u_i) \quad \forall \ i
\\
\sum_i a_i = 1
\end{gather}

\subsection{min}

The $\min$ function was implemented by multiplying the inputs and output of the $\max$ function by \num{-1}.

\begin{equation}
\min\left(a_1, a_2, ...\right) = -\max\left(-a_1, -a_2, ...\right)
\end{equation}

\subsection{Absolute value}

A linearized absolute value function was implemented by using Big M and introducing a new binary variable $u$ that takes a value of \num{1} when the input quantity $a$ is greater than zero. A new variable $z$ is introduced as the output of the absolute value function.

\begin{gather}
-M (1 - u) + a \le z \le M (1 - u) + a
\\
-M u - a \le z \le M u - a
\\
z \ge 0
\end{gather}

\subsection{Less than or equal to zero}

A linearized less-than-or-equal-to-zero function was implemented by using Big M and introducing a new binary output variable $z$. The input quantity is $a$.

\begin{equation}
-M z \le a \le M (1 - z)
\end{equation}

\subsection{Greater than or equal to zero}

A linearized greater-than-or-equal-to-zero logic function was implemented by multiplying the input to the less-than-or-equal-to logic function by \num{-1}.

\begin{equation}
\left(a \ge 0\right) = \left(-a \le 0\right)
\end{equation}

\subsection{Logical and}

A linearized logical $\logicand$ function was implemented by introducing a new binary variable $z$ for the output of the $\logicand$ function.
The input binary values are $a_1$ and $a_2$.
More than two input values were not considered in this formulation.

\begin{gather}
z \ge a_1 + a_2 - 1
\\
z \le a_i \quad \forall \ i
\end{gather}

\subsection{Logical or}

A linearized logical $\logicor$ function was implemented by introducing a new binary variable $z$ for the output of the $\logicor$ function.
The input binary values are $a_i$.

\begin{gather}
z \ge a_i \forall i
\\
z \le \sum_i a_i
\end{gather}
