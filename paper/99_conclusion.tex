\section{Conclusion}
\label{sec:conclusion}

This paper demonstrated a novel light-weight forecasting method for PV generation suitable for application to small agricultural microgrids.
The method utilized only basic weather forecast data and did not require large computing resources.
The proposed methods were compared to each other and to a reference persistence forecast.
The proposed methods gave lower MSE compared to the persistence forecast
over a test data set of
103 days.
Of the weather forecast sources that were compared,
SolCast cloudiness
obtained the lowest MSE.
Of the intra-day update methods that were compared,
the SARIMAX method
obtained the lowest MSE.
The forecast methods with the best MSE metrics were integrated into a simulated
agricultural microgrid with model-predictive control,
and the total system costs were compared to operation with persistence forecasts.
The costs for the system's simulated operation using persistence forecasts were somewhat better than the performance using the proposed method.
This unexpected result may warrant additional investigation.

One direction for future work is to extend the proposed lightweight method for PV power forecasting to provide probabilistic forecasts, quantifying the uncertainty in the forecast.
Such forecasts may then be used in stochastic model predictive control for the microgrid controls to make more robust energy planning decisions.
